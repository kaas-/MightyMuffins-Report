%!TEX root = ../master.tex
\chapter{Evaluation}\label{ch:evaluation}
This chapter will describe the evaluation process of the finished prototype. The chapter is spilt into three subsections: a planning section, actual evaluation description and analysis of evaluation results. 

\section{Further Context}\label{sec:furthercontext}
\todo{additional functionality: draw on image to change the sound} wut?

\section{Planning of evaluation}
This section will describe the thoughts the project group made before the actual evaluation test were made. The section will show a description of the evaluation tasks and question there were planned to be asked during the test. 

"underoverskrift eller noget" 

What do we want out of this test?

\begin{itemize}
\item Find out if the user can easily *øhm.. use / manurere the product
\item If the purpose of the product is clear to the user.
\item If the interface is easy to understand 
\item If there is anything to be redesigned / make more clear.

Interview question 
\begin{itemize}
\item How do you turn on the effect?
\item How do you turn off the effect?
\item What happend when you manipulated with the sliders?
\item Is the “help text” helpfull to you?
\item Is the help text clear?
\item Is the design pleasant to look at?
\item What do you think you are manipulating with?
\item What role has the painting in this?
\item Any other thoughts?

Assignments for the user to solve during the test 
\begin{itemize}
\item Turn on echo
\item Turn on echo and comb
\item Turn off comb
\item Turn on bandpass 
\item Turn everything off

Plan of setup *tegn billede af setup*
The test set up was planned to be held in a quiet area with one test participant at the time, located at Rendensburgade (The Create building) it was planned to conduct the test on at least 12 people (2 test people pr. group member) to get variation in the test results. 

Success criteria 
Does the product live up to our success criteria?

Technical evaluation 



\section{Evaluation test}


\section{Evaluation results}