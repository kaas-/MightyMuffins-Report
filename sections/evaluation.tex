%!TEX root = ../master.tex
\chapter{Evaluation}\label{ch:evaluation}
This chapter will describe the evaluation process of the finished prototype. The chapter is spilt into three subsections: a planning section, actual evaluation description and analysis of evaluation results. 

\section{Further Context}\label{sec:furthercontext}
\todo{additional functionality: draw on image to change the sound} wut?

\section{Planning of evaluation}
This section will describe the thoughts the project group made before the actual evaluation test were made. The section will show a description of the evaluation tasks and question there were planned to be asked during the test. 

"underoverskrift eller noget" 

What do we want out of this test?

\begin{itemize}
\item The user can alter the output by applying filters.
\item The user understands the interface  
\item The prototypes usability is over 80 percent of the users satisfaction.
\end{itemize}

Interview question 
\begin{itemize}
\item What did you experience when manipulating with the sliders?
\item did you 
\item Where you confused about the design?
\item was the “text” helpfull to you?
\item Any other thoughts?
\end{itemize}

Assignments for the user to solve during the test 
\begin{itemize}
\item Turn on echo
\item Turn on echo and comb
\item Turn off comb
\item Turn on bandpass 
\end{itemize}

Plan of setup *tegn billede af setup*
The test set up was planned to be held in a quiet area with one test participant at the time, located at Rendensburgade (The Create building) it was planned to conduct the test on at least 12 people (2 test people pr. group member) to get variation in the test results. 

Success criteria 
Does the product live up to our success criteria?

Technical evaluation 



\section{Evaluation test}


\section{Evaluation results}