%!TEX root = ../master.tex
\chapter{Conclusion}\label{ch:conclusion}
The purpose of this project was to create an interface for controlling filters and applying them to a real-time audiolisation of an image. The prototype interpreted the different colour channels of an RGB image as individual frequencies, each of which had a different filter applied to it. Each filter had a coefficient that was tied to a slidepotentiometer, that the user could control. The three filters used for testing were a comb filter, a bandpass filter, and a highshelf filter. Through testing it was shown that the audiolisation method produced distinct pieces of audio. The bandpass and highshelf filters altered the sound in ways that users could recognize, but the comb filter did not. Usability tests showed that the interface was usable, but that the use of technical language as labels was a cause of usability issues. In conclusion, the product was successful.
