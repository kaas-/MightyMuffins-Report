%!TEX root = ../master.tex
\chapter{Introduction}\label{ch:introduction}
\todo{Since our project is a creation of a tool, our introduction should reflect this when we need to write it. The introduction from 'An Experimental System for Auditory Image Representation'}

The purpose of this project is to create an experimental system for auditory image representation. This will be used as an innovative tool for everyone with artistic interest, which is to be expanded upon, in terms of creating new representations of visual perception. The report will identify previously experimented prototypes for audiolisation of static images and currently used audio software including the methods used in audio processing. These will lead to a series of success criteria which are to evaluate the projects developed solution. To evaluate these criterias, a series of testing and quantative statistics will be performed in order to conclude on the prototypes performance.   


\todo {Denne del kan enten slettes eller tilføjes.}Through this report, there will be a study with previously experimented prototypes that used interactive audiolisation of static images and a technical replacement for people with visual deficits. This includes currently used audio software and methods in audio processing such as filters and spectrograms. Taking these This will lead to the designing of a potential prototype based on the accomplished research and testing of different interfaces looking at the users feedforward, feedback and perceived affordance. With an established interface in the users preference, a       

Key points:

You need to define what you are going to talk about. Otherwise your marker can't tell if you've talked about it meaningfully or not.
You need to show your marker what you are trying to do with your topic - your direction.
You need to show your marker what you are going to cover (and what you're not, if need be).
You need to give your marker background information necessary to their understanding.