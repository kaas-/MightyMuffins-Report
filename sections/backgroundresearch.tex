%!TEX root = ../master.tex
\chapter{Background Research}\label{ch:bgresearch}



\section{Types of sources}\label{sec:typesofsources} 
The sources used in this chapter include scientific articles regarding the topic “Image to sound conversion”. The articles are published in the fields of physics, sound art and digital media. Other sources include recorded lectures of one of the articles authors.

\section{Previous work}\label{sec:previouswork}

\subsection{An experimental system for auditory image representation}\label{sec:experimentalsystem}
\todo{make introduction, then go into blindess}
Blindness has consequences for an individual's perception of the environment. An experimental system for vision substitution was developed by Peter B. L. Meijer. The system consists of a computer connected to a camera, which records real-time images and converts them into sound. 

The system used a method called time-multiplexed mapping, where the distribution of rows and columns in an image, the height (M) and width (N) respectively, where the pixels are stored in a matrix. The time spent scanning the image (R)is used to define when the current image ends and the next image begins


The time (R) of scanning the image runs from the begining of the image and stops when the previous image ends. An example of this method is seen in figure \todo{reference here!}. 

(image)
  
The images have a resolution of 64 * 64 pixels with 16 gray-tones per pixels.  

The experiment showed promising functionality to convert images to sound but lacks a field study test on people with blindness. Moreover, the advantages and disadvantages of this system is yet to be proved. This questions the reliability of the system since there has been no recordings of testing data performed during the experimentation. However theory supports the system's functionality. \todo{the math can be used in other projects}

\subsection{The sound of Photographic Image}\label{sec:soundarticle}
\todo{most focus on point relevant for the project. please rework}

A use of photographic images for conversion into sound was performed by the chair of digital media and director of culture lab at newcastle university, Atau Tanaka. 


An article that describes six years of a sound artist's work  who converts already existing pictures into soundwaves. In the article the artists describes the inspiration of the three works and the technical methods used to create the sound of the works. Two of the three art works was released on CD while the third work was an interactive art installation not only showing audio but also chaing images. The methods used in one of his works (fig 1) uses temporary mapping and additive synthesis to convert the raw grayscale image into a sound file scanning the picture from left to right row by row, to create variations in the sound, every pixel is scanned and if a pixel is bright it will create high notes.  Tanaka also used sonograms and infrared cameras to create different looking pictures with variating sounds, and an opportunity for the viewer to interact with the image by moving in front of it so their silhouettes would appear on the screen. The result of Tankas three works can be seen in the three figures below, as mentioned, two of the works ( fig 1. 9m14s over Vietnam and fig 2. Bondage) was released on CD.


- No evaluation performed

- Exhibition in paris for audience, no comments on the exhibit

- Used shoji - wooden structure with panels displaying images which the viewer could interact with by the use of their silhouettes. 
 

\section{Methods used to evaluate}\label{sub:methodsusedtoevaluate}






\section{State of the art}\label{sec:stateart}

\subsection{Sonic Photo}\label{sub:sonic}
\todo{make into reference}
Can be found here: http://www.skytopia.com/software/sonicphoto/

Sonic photo is a program that allows the user to transform an image into sound. The user can load an image into the program, and adjust various parameters to modify the resulting audio. The parameters include  frequency(Hz), brightness, tone, harmony quantization, etc.

\todo{more general information about the program. Which points to we want to make, by writing about this program?}