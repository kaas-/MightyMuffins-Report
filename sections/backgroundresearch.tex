%!TEX root = ../master.tex
\chapter{Background Research}\label{ch:bgresearch}



\section{Types of sources}\label{sec:typesofsources} 
The sources found during the background research phase consists of two scientific articles inside the topic “Image to sound converting” the articles are written by differents researchers who work in the field of physics, sound art and digital media. Other types of sources used to find previous work consists of three short videos found on YouTube through one of the writes webpage. 


\section{Previous work}\label{sec:previouswork}

\subsection{The sound of Photographic Image}\label{sec:soundarticle}
An article that describes six years of work of a sound artist who converts already existing pictures into soundwaves. In the article the artists describes the inspiration of the three works and the technical methods used to create the sound of the works. Two of the three art works was released on CD while the third work was an interactive art installation not only showing audio but also chaing images. The methods used in one of his works (fig 1) uses temporary mapping and additive synthesis to convert the raw grayscale image into a sound file scanning the picture from left to right row by row, to create variations in the sound, every pixel is scanned and if a pixel is bright it will create high notes and vice versa.  Tanaka also used sonograms and infrared cameras to create different looking pictures with variating sounds, and an opportunity for the viewer to interact with the image by moving in front of it so their silhouettes would appear on the screen. The result of Tankas three works can be seen in the three figures below, as mentioned, two of the works ( fig 1. 9m14s over Vietnam and fig 2. Bondage) was released on CD.



\section{Methods used to evaluate}\label{sub:methodsusedtoevaluate}






\section{State of the art}\label{sec:stateart}

\subsection{Sonic Photo}\label{sub:sonic}
Can be found here: http://www.skytopia.com/software/sonicphoto/

Sonic photo is a program that allows the user to transform an image into sound by loading an image into the program, from where various parameters can be modified to alter the image and sound. 

It's possible to alter the frequency(Hz), brightness, tone, and the harmony quantization. 

through the use of the instrument and harmony quantization it's possible to alter the sound which the picture produces. 
You can decide what harmonics and chords you want to use. 
it's also possible to alter the harmonics brightness, the strength of the quantization and the base pitch that the picture produces. 