%!TEX root = ../master.tex
\chapter{Implementation}\label{ch:implementation}

\section{Tools used}\label{sec:codeoverview}
The tools used in this project is Pure data and Arduino. 
Pure data is a C/C++ based coding language which utilises pre-constructed blocks of code which have to be connected to one another. Pure data is the code implemented in our project which is responsible for the audio and image processing. Gem is used for the image processing purpose, which is a library in Pura data that allows for various image processing and animation methods.  
Furthermore three different filters have been constructed and implemented using pure data.

Arduino is a micro controller that allows the user to code and create their own circuits using various components, this allows for a wide range of purposes. In this project the Arudino is responsible for the physical interface which the user has available. 

\section{Code overview}\label{sec:codeoverview}
	\subsection{Image processing}\label{sub:imageprocessing}
	As the project revolves around pictures being audiolised, various image processing methods have been implemented. To get an image loaded into Pure data the library gem have been imported and applied. In gem it is possible to go through each pixel getting its RGB value as three different values. This is done using a double nested for-loop that is constructed using the expr function in Pure data, which allows for construction of consecutive if-statements. The first if-statement gives an output that is increased by 0.001 for every cycle if the input value is less than one. The second if-statement checks if the input value is greater or equal to one, if that is the case it adds 0.01 to a value starting at zero, which is then outputted. The last if-statement checks if the second input is greater or equal to 1, if that is the case subtract 1 from the value of the second input.
	
	Pipe is an object that delay the input given to it by a specific amount of milliseconds. It is used here to make sure that the program does not give a stack overflow error, and that it doesn't process the pixels too fast. By default in the program the pipe is set to 50 milliseconds since this allows the program to smoothly go through each pixel one by one. 
	When the pixel data is being processed in the program, it outputs the value of the three colour channels in the specific pixel, meaning that the output consists of three different value representing the R, G and B value. With the three RGB values it is possible to plot the values giving a graph of how the colour distribution is in the picture, but it is also possible to normalize the values into a signal by normalizing them from zero to one into negative one to positive one. By doing this the values have now been turned into a signal, which in terms should be able to produce a sound.

	\subsection{Audiolisation of image}\label{sub:audiolisationofimage} 
	Audiolisation of an image means to turn the image into some kind of audio, which is what the program explained in \ref{sub:imageprocessing} is programmed to do. 

	\subsection{Filters}\label{sub:filters}
	
	\subsection{Arduino}\label{sub:arduino}
	
\section{Physical interface}\label{sec:physicalinterface}
	\subsection{Components}\label{sub:components}
	
	\subsection{Circuit diagram}\label{sub:circuitdiagram}