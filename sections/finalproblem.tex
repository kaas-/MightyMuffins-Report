%!TEX root = ../master.tex
\chapter{Final Problem Statement}\label{ch:finalproblem}
\todo{write a paragraph that will summaries the points that made us come to this problem statement, and the gap we will investigate}

The user group for this project are creative people of all ages and gender, as creativity is not limited by age nor gender. 
The project is aiming towards creating a creative tool to be used as an additional method of expression. 

The final problems statement for this project is as following:
\textit{How can an interface be applied to a real-time audiolisation of an image that enables control of filters in a user-friendly manner}

In order to fulfill the final problem statement there needs to be established some success criteria which can be validated through testing of the prototype. 
\begin{enumerate}
\item The user has to be able to alter the audio effect
\item The image has to be audiolised 
\item The usability has to be above 80 percent
\end{enumerate}

!anna!
To obtain a successful prototype, the problem formulation has to be answered as well as the success criteria has to be fulfilled. It is therefore important that the prototype can convert an image into a sound, for then enabling the possibility to apply audio filters which the user can alter through a physical interface. The usability has to be over 80\% in order to ensure an adequate understanding of the interface. The usability will be tested with a satisfaction test using a likert scale.


!victor!
For this project it is important that the prototype is able to convert an image into a frequency in order to apply filters which the user can alter through audio effects to have a successful prototype. This should be observation able in a computer software program that establish a frequency span from the image pixels which then can be altered through the interface. The reason for a usability that should be above 60 percent is required to know if the user understands the interface and data collected from testing can be processed through quantitative analysis. This data is based on a satisfaction scale of 1 to 10 and can help in the later development of this project. 

