%!TEX root = ../master.tex
\chapter{Final Problem Statement}\label{ch:finalproblem}
\todo{write a paragraph that will summaries the points that made us come to this problem statement, and the gap we will investigate}

The user group for this project are creative people of all ages and gender, as creativity is not limited by age nor gender. The project is aiming towards creating a creative tool to be used as an additional method of expression. 
The final problems statement for this project is as following:



\textit{How can an interface be applied to a real-time audiolisation of an image that enables control of filters in a user-friendly manner}



In order to fulfill the final problem statement there needs to be established some success criteria which can be validated through testing of the prototype. 
\begin{enumerate}
\item The image has to be audiolised 
\item The user has to be able to alter the audio effect 
\item The product should be deemed usable through qualitative testing
\end{enumerate}

To obtain a successful prototype, the problem formulation has to be answered as well as the success criteria has to be fulfilled. It is therefore important that the prototype can convert an image into a sound, for then enabling the possibility to apply audio filters which the user can alter through a physical interface. The first two success criteria can be reached by using previous methods used to audiolise an image, as well as constructing an interface which utilises the knowledge of electrical components obtained from the course Physical Interface Design.
To ensure that the user has an adequate understanding of the interface, the final design needs to be tested through a semi-structured interview where the user interacts with the system by attempting to complete various tasks. After the prototype has been tested, an evaluation will be conducted on the qualitative data which will validate if the product is deemed usable.