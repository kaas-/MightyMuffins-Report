%!TEX root = ../master.tex
\pdfbookmark[0]{Title page}{label:titlepage_en}
\aautitlepage{%
  \englishprojectinfo{
    The Use of Image Audiolisation as a Creative Tool %title
  }{%
    Sound Computing and Sensor\\
    Technology %theme
  }{%
    Autumn Semester 2016 %project period
  }{%
    MTA16430 % project group
  }{%
    %list of group members
    Anna Harbo Søndergaard\\
	Camilla Marie Skytte\\
	Markus Kristian Ørbæk Bertelsen\\
	Peter Kejser Jensen\\
	Rasmus Kaasgaard Christiansen\\
	Victor Andreas Graffmann

  }{%
    %list of supervisors
    Knud Bank Christensen
  }{%
    1 % number of printed copies
  }{%
    \today % date of completion
  }%
}{%department and address
  \textbf{Media Technology}\\
  Aalborg University\\
  \href{http://www.aau.dk}{http://www.aau.dk}
}{% the abstract
  This project explores the use of image audiolisation as a creative tool. Through an iterative design process a prototype is created which audiolises an image and applies filters to the resulting audio. These filters are manipulated using a physical interface. The usability of the interface is tested using semi-structured interviews. Through an analysis of quantitative data it is shown that the interface is usable, the audiolisation produces a distinct sound, and that most of the effects can be discerned. 
}
