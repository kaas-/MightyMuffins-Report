%!TEX root = ../master.tex
\chapter{Discussion}\label{ch:discussion}
This chapter contains discussion of the entire project 

\section{Sources of error}

\todo{ ved ikke hvor jeg skal skrive det her, men det kommer lige her, skal sættes ind det rette sted og rettes i}
 Since the final test were only conducted on 14 medialogy students and one Art \& technology student, is it hard to say if our product would be just as "easy to understand" for a normal museum visitor or any other user. To investigate if the product would be accepted among people with no knowledge of audio processing or people who are having trouble understanding "techical stuff I don't know" the optimal solution would be to test the product in context, in this case it could be at an art museum, where the test would be conducted on guests of the museum which could be people of all age groups. 
 
 \section{Contex of use}
 - During this project, two of the members of the project group visited the art museum Kunsten in Aalborg. The goal was to find out how the artefact could be used in context at the museum. 
 
 - how would the setup be so the visitors could draw a connection between the painting and the artefact 
 
 - pictures from Kunsten 
 
 - Results from the visit
 
 - show sketch of grandma at the art museum / solution. 
 
 - One of the test participants was unsure of the connection of the artefact and the paintings shown during the test, he needed to be explained that the sound was made by the painting.
 
 - maybe make a sketch of how the problem could be solved / show sketches of the sound waves on the grandma picture 
 
\section{Wider context}
What can this be used for?

It would be possible to alter the function of the project, such that the user would have the possibility to decide where in the picture the program should be looking, as this would allow the user further freedom over the creation of sound, made by the product. But if it wasn't implemented correct, it would add to the confusion of the user, due to the amount of extra components on the interface. It could be done using an array of various components, such as a controller, slide potentiometers, or rotary encoders. 

Instead of choosing where in the picture to look, instead make it so that the user is able to create the picture being utilised by the software. This could allow the user to differentiate themselves from each other, and create unique and interesting sounds, increasing its potential as a creative tool for artists as well as non artists. This could also make it an entertaining product to have at things like festivals, events, or at a museum. 

An alternative version of the product, could be if instead of a picture, it worked with a video feed, allowing it to render and manipulate the information given in real-time. Allowing the user to translate his surrounding to sounds, giving a new perspective on the world, if this could be implemented onto a hand-held device, the user would be able to use things such as mobile phones or tablets. Which would allow the user to translate various areas without the need for a big machinery. 
The user could also have the possibility to upload their own pictures, videos, or even add their own sounds which are played when translating. 

