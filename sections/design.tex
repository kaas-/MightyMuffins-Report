%!TEX root = ../master.tex
\chapter{Design}\label{ch:design}
\todo{This part needs rework. Write after chapter is done.}
This chapter will cover the design iterations and the decisions that were made. 


\section{Peripheral devices}
The user has to be able to affect a pre-existing audio file by interacting with a physical interface. The system with apply an affect which can be modified through a physical scalar which changes the variable range of the power of the effect from 0 PERCENT to 100 PERCENT, or in digits 0 to 1. For this interface these different variations of a physical scalar has been considered: 

\begin{itemize}
\item Pedal
\item Joystick
\item Variable resistor
\item Bend sensor
\end{itemize}


\todo{talk about what is the prefence of different interfaces for the user}

Each of these peripheral devices makes the user interact with the prototype differently and changes the function of each device. 

The way the joystick works is by incorporating 2 axis on the joystick, such the y-axis is equivalent to an axis made in the software which could control the intensity of the effect, 
\todo{Markus: What did you mean with this exactly?}
where the x-axis would affect the effect itself. 

As for the variable resistor, it is a knob the user can turn, which then affected the output, depending on what resistance there is in the electronic system. The Arduino will register measured change in resistance and send back a signal to the software in order to change the effect.

\todo{Isnt this the same as the variable resistor? why are there two catagories when they are the same}
The bend sensor will function similar to the variable resistor, as the bend sensor varies in resistance depending on how much it is bend.

The way that these devices would be used, are as controllers for the output effect. Depending on what device is chosen, the integration for the user will be different. This will be tested in the iterations of the prototype and discussed further to make sure that the interface is user-friendly and easy to understand. 

\todo{this needs to be discussed}
However, since this is a tool for creative inspiration, it is also important to give the user the possible to modify the product a bit, to what they are trying to make. As for a painter, he would maybe change the picture, where a musician would play around with the sounds that are produced. 