%!TEX root = ../master.tex
\chapter{Design}\label{ch:design}

\section{Peripheral devices}
As the user has to be able to affect the audio produced by the project, it's important that they have some kind of interface to interact with. For this interface different methods have been considered such as:

\begin{itemize}
\item Pedal
\item Joystick
\item Variable resistor
\item Bend sensor
\item etc.
\end{itemize}

Each of these peripheral devices, makes the user interact with the product differently. Like the pedal makes the user press a pedal and depending on the angle of the pedal, it gives back a different value, which gives a different kind of effect.

The way the joystick would work, is by incorporating 2 axis on the 2 axis on the joystick, such that the y-axis on the joystick is equivalent to an axis made in the program, which could control intensity of the effect, where the x-axis would affect the effect itself. 

As for the variable resistor, it would be like a knob which the user could turn, which then affected the output, depending on what resistance there is in the electronic system which would be registered on the Arduino which would then send back a signal to change the effect. 

The bend sensor would function much like the variable resistor, as the bend sensor varies in resistance depending on how much it is bend.